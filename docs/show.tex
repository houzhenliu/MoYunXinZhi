\documentclass[a4paper, 12pt]{article}
\usepackage[UTF8]{ctex}
\usepackage{graphicx}
\usepackage{hyperref}
\usepackage{listings}
\usepackage{color}
\usepackage{geometry}
\usepackage{tcolorbox}
\usepackage{enumitem}
\usepackage{fontawesome5}
\usepackage{xcolor}

\geometry{a4paper,left=2.5cm,right=2.5cm,top=2.5cm,bottom=2.5cm}

% 定义JSON语言支持
\lstdefinelanguage{json}{
    keywords={true,false,null},
    keywordstyle=\color{blue},
    stringstyle=\color{codestring},
    string=[s]{"}{"},
    comment=[l]{:},
    commentstyle=\color{codekey},
    morecomment=[l]{:},
    morestring=[b]',
    basicstyle=\ttfamily\small,
}

% 定义JavaScript语言支持
\lstdefinelanguage{javascript}{
    keywords={break, case, catch, continue, debugger, default, delete, do, else, false, finally, for, function, if, in, instanceof, new, null, return, switch, this, throw, true, try, typeof, var, void, while, with, let, const, class, export, import, super, extends},
    keywordstyle=\color{blue},
    ndkeywords={class, export, boolean, throw, implements, import, this},
    ndkeywordstyle=\color{blue},
    identifierstyle=\color{black},
    sensitive=false,
    comment=[l]{//},
    morecomment=[s]{/*}{*/},
    commentstyle=\color{green!60!black},
    stringstyle=\color{codestring},
    string=[b]',
    string=[b]"
}

% 定义bash语言支持
\lstdefinelanguage{bash}{
    keywords={if, then, else, elif, fi, for, while, in, do, done, exit, return, function},
    keywordstyle=\color{blue},
    sensitive=true,
    comment=[l]{\#},
    commentstyle=\color{green!60!black},
    stringstyle=\color{codestring},
    string=[b]",
    string=[b]'
}

\definecolor{codebackground}{rgb}{0.95,0.95,0.95}
\definecolor{codestring}{rgb}{0.6,0.1,0.1}
\definecolor{codekey}{rgb}{0.1,0.1,0.6}

\hypersetup{
    colorlinks=true,
    linkcolor=blue,
    filecolor=magenta,
    urlcolor=cyan,
}

\lstset{
    basicstyle=\ttfamily\small,
    backgroundcolor=\color{codebackground},
    keywordstyle=\color{blue},
    stringstyle=\color{codestring},
    commentstyle=\color{green!60!black},
    breaklines=true,
    breakatwhitespace=true,
    frame=single,
    showstringspaces=false,
    tabsize=2,
    captionpos=b
}

\title{\LARGE{\textbf{新闻稿生成器系统使用手册}}}
\author{项目组:ymxz \\ 维护者:houzhenliu}
\date{\today}

\begin{document}

\begin{titlepage}
    \centering
    \vspace*{1cm}
    % 移除SVG图片,LaTeX不直接支持SVG格式
    \textsc{\LARGE 挑战杯作品:“墨韵新智”新闻稿生成器系统}\\[0.5cm]
    \rule{\linewidth}{0.5mm} \\[0.4cm]
    { \huge \bfseries 新闻稿生成器\\系统使用手册 \\[0.4cm] }
    \rule{\linewidth}{0.5mm} \\[1.5cm]
    
    \begin{minipage}{0.5\textwidth}
        \begin{center}
            \large
            \emph{发布日期:} 2025年8月13日
        \end{center}
    \end{minipage}
    
    \vfill
    
    {\large 项目组:墨韵新智项目组\\
    维护者:刘厚振,赵瀚焜\\
    \today}
    
\end{titlepage}

\tableofcontents
\newpage

\section{项目概述}

\subsection{系统简介}
墨韵新智是一个使用 Next.js 和 Material-UI 构建的现代化前端应用程序,旨在帮助用户快速生成专业的新闻稿。用户可以通过上传 JSON 数据文件或直接在界面中输入信息,系统将调用后端 API 自动生成符合专业标准的新闻稿内容。

\subsection{核心功能}
\begin{itemize}
    \item \textbf{文件上传}:支持拖拽和点击上传 JSON 文件
    \item \textbf{AI 生成}:调用后端 API 生成专业新闻稿
    \item \textbf{实时预览}:在线查看生成的新闻稿内容
    \item \textbf{下载功能}:一键下载生成的新闻稿文件(支持多种格式)
    \item \textbf{现代化界面}:使用 Material-UI 构建的美观界面
    \item \textbf{响应式设计}:支持各种设备屏幕尺寸
\end{itemize}

\section{技术架构}

\subsection{技术栈}
\begin{itemize}
    \item \textbf{前端框架}:Next.js 15.4.5
    \item \textbf{UI 库}:Material-UI (MUI) 7.2.0
    \item \textbf{开发语言}:TypeScript
    \item \textbf{样式}:Tailwind CSS + Material-UI
    \item \textbf{图标}:Material Icons
    \item \textbf{Markdown 渲染}:react-markdown + remark-gfm
    \item \textbf{文档生成}:docx
\end{itemize}

\subsection{系统架构图}
\begin{tcolorbox}[colback=blue!5,colframe=blue!40,title=系统架构图]
\centering
用户界面 (Next.js) $\longrightarrow$ API 路由 $\longrightarrow$ 第三方 AI 服务 \\
$\uparrow$ \hspace{6cm} $\downarrow$ \\
输出展示与下载 $\longleftarrow$ 数据处理 $\longleftarrow$ 响应数据
\end{tcolorbox}

\subsection{项目结构}
\begin{lstlisting}[language=bash, caption=项目文件结构]
src/
├── app/
│   ├── api/
│   │   ├── generate-news/
│   │   │   └── route.ts      # 新闻生成 API 路由
│   │   └── upload-file/
│   │       └── route.ts      # 文件上传 API 路由
│   ├── globals.css           # 全局样式
│   ├── layout.tsx            # 应用布局
│   └── page.tsx              # 主页面
├── components/
│   ├── NewsGeneratorApp.tsx  # 主要应用组件
│   └── NewsInputComponent.tsx # 新闻输入组件
public/
├── sample-news-data.json     # 示例数据文件
└── ...                       # 其他静态资源
\end{lstlisting}

\section{安装与配置}

\subsection{环境要求}
\begin{itemize}
    \item Node.js 18+
    \item npm 或 yarn
    \item 互联网连接(用于API调用)
\end{itemize}

\subsection{安装步骤}
\begin{enumerate}
    \item \textbf{克隆项目代码}
    \begin{lstlisting}[language=bash]
    git clone https://github.com/houzhenliu/ymxz.git
    cd ymxz
    \end{lstlisting}
    
    \item \textbf{安装依赖}
    \begin{lstlisting}[language=bash]
    npm install
    # 或
    yarn install
    \end{lstlisting}
    
    \item \textbf{配置环境}
    \begin{lstlisting}[language=bash]
    # 复制配置文件示例并进行编辑
    cp config.json.example config.json
    # 使用文本编辑器打开并填写必要的配置信息
    \end{lstlisting}
\end{enumerate}

\subsection{配置文件说明}
系统使用 \texttt{config.json} 文件存储关键配置参数。以下是配置文件的结构和参数说明:

\begin{lstlisting}[language=json, caption=config.json 示例]
{
  "API Key": "your_api_key_here",
  "API Secret": "your_api_secret_here",
  "API flowid": "your_flow_id_here",
  "url": "https://xingchen-api.xf-yun.com/workflow/v1/chat"
}
\end{lstlisting}

\begin{itemize}
    \item \texttt{API Key}:API 访问密钥,由服务提供方提供
    \item \texttt{API Secret}:API 访问密钥,由服务提供方提供
    \item \texttt{API flowid}:工作流 ID,用于标识特定的生成流程
    \item \texttt{url}:API 服务的访问地址
\end{itemize}

\section{使用指南}

\subsection{启动系统}
\begin{lstlisting}[language=bash]
# 开发模式启动
npm run dev

# 生产模式构建和启动
npm run build
npm start
\end{lstlisting}

应用将在 \url{http://localhost:3000} 启动。

\subsection{基本操作流程}
\begin{enumerate}[label=\arabic*.]
    \item \textbf{访问系统}:打开浏览器,访问 \url{http://localhost:3000}
    \item \textbf{上传数据}:通过拖拽或点击上传区域选择 JSON 文件
    \item \textbf{编辑内容}:根据需要修改或补充内容
    \item \textbf{生成新闻稿}:点击"生成新闻稿"按钮
    \item \textbf{查看预览}:在页面下方实时预览生成的内容
    \item \textbf{下载文件}:点击"下载"按钮保存生成的新闻稿,可以生成word或者txt格式
\end{enumerate}

\subsection{数据格式要求}
系统接受符合以下格式的 JSON 数据:

\begin{lstlisting}[language=json, caption=标准输入数据格式]
{
  "AGENT_USER_INPUT": "新闻主题或描述",
  "live": [
    { "url": "图片URL", "desc": "图片描述" }
  ],
  "quote": [
    {
      "name": "引述人姓名",
      "image": "引述人照片URL",
      "content": "引述内容",
      "needAiSummary": 0,
      "desc": "图片描述",
      "customDesc": "自定义描述"
    }
  ]
}
\end{lstlisting}

\subsection{功能说明}

\subsubsection{文件上传}
系统支持两种上传方式:
\begin{itemize}
    \item 拖拽文件到指定区域
    \item 点击上传区域,通过文件选择器选择文件
\end{itemize}

上传文件后,系统会自动验证 JSON 格式并显示文件内容。

\subsubsection{新闻稿生成}
点击"生成新闻稿"按钮后,系统会:
\begin{itemize}
    \item 发送请求到后端 API
    \item 显示生成进度
    \item 完成后在预览区显示生成结果
\end{itemize}

\subsubsection{结果导出}
系统支持多种导出格式:
\begin{itemize}
    \item 文本格式(.txt)
    \item Word 文档(.docx)
\end{itemize}

点击对应的下载按钮即可导出相应格式的文件。

\section{API 接口说明}

\subsection{新闻生成接口}
\begin{tcolorbox}[colback=gray!5,colframe=gray!50,title=API 路径]
\textbf{POST} /api/generate-news
\end{tcolorbox}

\subsubsection{请求参数}
请求体包含符合前述数据格式的 JSON 对象。

\subsubsection{响应格式}
\begin{lstlisting}[language=json]
{
  "content": "生成的新闻稿内容",
  "title": "新闻稿标题(可选)",
  "summary": "新闻稿摘要(可选)"
}
\end{lstlisting}

\subsection{文件上传接口}
\begin{tcolorbox}[colback=gray!5,colframe=gray!50,title=API 路径]
\textbf{POST} /api/upload-file
\end{tcolorbox}

\subsubsection{请求参数}
请使用 FormData 格式提交文件,并在请求头中包含必要的授权信息:
\begin{lstlisting}[language=javascript]
headers: {
  'Authorization': 'Bearer [token]'
}
\end{lstlisting}

\subsubsection{响应格式}
响应将直接转发自上游服务。

\section{常见问题}

\subsection{配置相关}
\begin{tcolorbox}[colback=red!5,colframe=red!40,title=问题:配置文件缺失]
\textbf{错误信息}:配置文件读取失败,请检查 config.json 文件是否存在且格式正确

\textbf{解决方案}:
\begin{itemize}
    \item 确认在项目根目录中存在 config.json 文件
    \item 按照 config.json.example 的格式检查内容
    \item 确保 JSON 格式无误(无多余逗号、引号匹配等)
\end{itemize}
\end{tcolorbox}

\begin{tcolorbox}[colback=red!5,colframe=red!40,title=问题:API 配置缺失]
\textbf{错误信息}:API 配置缺失,请检查 config.json 文件中的配置信息

\textbf{解决方案}:
\begin{itemize}
    \item 确认已填写 API Key、API Secret 和 API flowid
    \item 联系服务提供商获取有效的 API 凭证
\end{itemize}
\end{tcolorbox}

\subsection{使用相关}
\begin{tcolorbox}[colback=yellow!5,colframe=yellow!40,title=问题:文件上传失败]
\textbf{可能原因}:
\begin{itemize}
    \item 文件格式不是有效的 JSON
    \item 文件大小超出限制
    \item 网络连接问题
\end{itemize}

\textbf{解决方案}:
\begin{itemize}
    \item 使用 JSON 验证工具检查文件格式
    \item 确保文件大小合理
    \item 检查网络连接
\end{itemize}
\end{tcolorbox}

\begin{tcolorbox}[colback=yellow!5,colframe=yellow!40,title=问题:新闻稿生成失败]
\textbf{可能原因}:
\begin{itemize}
    \item API 服务不可用
    \item 提交的数据格式有误
    \item 授权失败
\end{itemize}

\textbf{解决方案}:
\begin{itemize}
    \item 检查 API 服务状态
    \item 检查提交的数据格式
    \item 验证 API 凭证是否有效
\end{itemize}
\end{tcolorbox}

\section{许可与支持}

\subsection{许可证}
本项目采用 MIT 许可证。详情请参阅项目根目录中的 LICENSE 文件。

\subsection{支持与贡献}
\begin{itemize}
    \item \textbf{问题报告}:通过 GitHub Issues 提交问题
    \item \textbf{功能请求}:通过 GitHub Issues 提交功能需求
    \item \textbf{代码贡献}:欢迎提交 Pull Request
\end{itemize}

\subsection{联系方式}
\begin{itemize}
    \item 项目仓库:\url{https://github.com/houzhenliu/ymxz}
    \item 维护者:houzhenliu
\end{itemize}

\vspace{1cm}
\begin{center}
\rule{0.5\linewidth}{0.5pt}\\
\textbf{文档结束}
\end{center}

\end{document}
